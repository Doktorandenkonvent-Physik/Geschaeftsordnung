\documentclass[a4paper, parskip=half, numbers=noenddot]{scrartcl}

\usepackage{a4wide}
\usepackage{geometry}
\geometry{left=25mm, right=25mm, top=23mm, bottom=33mm}

\usepackage[ngerman]{babel}
\usepackage[T1]{fontenc}
\usepackage[utf8x]{inputenc}

\usepackage{microtype}

\usepackage[juratotoc,ref=parlong]{scrjura}

\usepackage{hyperref}

% setzt einen kleinen Abstand \, zwischen Zahl und Buchstabe bei Paragraphen; ist so gewünscht
\renewcommand*{\thecontractSubParagraph}{%
{\theParagraph\,\alph{contractSubParagraph})}}


\title{Geschäftsordnung des Konvents der Doktorandinnen und Doktoranden an der KIT-Fakultät für Physik}
% \author{Version 1.0}
\date{Stand: 27. März 2015}
\hypersetup{
    pdftitle={Geschäftsordnung des Konvents der Doktorandinnen und Doktoranden an der KIT-Fakultät für Physik},
    pdfauthor={Konvents der Doktorandinnen und Doktoranden an der KIT-Fakultät für Physik},
    pdfborder={0 0 0.5},
    linkbordercolor={0 0.61 0.50}
}


\begin{document}

\maketitle
% \thispagestyle{empty}

% \pagestyle{empty}
% \newpage
% % Für Printversion eine leere Seite
% \rule{0mm}{0mm}
% \newpage


\begin{contract}

% \setcounter{page}{1}
% \pagestyle{plain}

\tableofcontents
% \newpage


\Paragraph{title = Aufgaben}
\label{aufgaben}

Der Konvent der Doktorandinnen und Doktoranden ist die gemäß § 38 Absatz 7
Landeshochschulgesetz (LHG) gebildete Interessenvertretung der Doktorandinnen und
Doktoranden des Karlsruher Institut für Technologie (KIT).
Der Konvent ist die Basis für den Dialog zwischen den Organen des KIT und den Doktorandinnen und Doktoranden.
Der Konvent berät die die Doktorandinnen und Doktoranden betreffenden Fragen und spricht Empfehlungen an die Organe des KIT aus.
Der Konvent ist Ansprechpartner für alle Doktorandinnen und Doktoranden an der KIT-Fakultät Physik und deren Belange.
Entwürfe für Promotionsordnungen werden dem Konvent zur Stellungnahme zugeleitet.
Der Konvent kommuniziert bei Fragen, die Doktorandinnen und Doktoranden KIT-weit betreffen, mit den anderen Konventen.

\Paragraph{title = Mitglieder}

Mitglieder des Konvents sind alle von der KIT-Fakultät für Physik zur Promotion angenommenen Doktorandinnen und Doktoranden. 

\Paragraph{title = Vorstand}

Der Vorstand des Konvents besteht aus insgesamt 5 Mitgliedern. Durch diese sollten alle
Gruppen an der Fakultät vertreten sein.
Die Aufgaben des Vorstands sind:
\begin{enumerate}
	\item Einberufung und Leitung der Sitzungen des Konvents
	\item Vertretung des Konvents gegenüber KIT-Gremien und nach außen
	\item Kommunikation mit anderen Doktorandenkonventen inner- und außerhalb des KIT
	\item Entsendung eines Sprechers als Gast in den Sitzungen des Fakultätsrats
\end{enumerate}
Der Vorstand ist befugt im Namen des Konvents zu sprechen und abzustimmen.
Der Vorstand ist an die Beschlüsse des Konvents gebunden und leitet diese an Gremien weiter.

Die Mitglieder des Vorstands werden von den anwesenden Mitgliedern des Konvents gewählt. Für
die Wahl hat jedes Konventsmitglied Stimmen entsprechend der Anzahl der Vorstandsmitglieder. Stimmen können nicht kumuliert werden.
Gewählt sind die Personen, die die meisten Stimmen auf sich vereinen können. Bei Stimmengleichheit entscheidet das Los. Die Wiederwahl der Vorstandsmitglieder ist zulässig.
Der Vorstand wählt mit einfacher Mehrheit der Stimmen aus seiner Mitte eine Vorsitzende bzw.
einen Vorsitzenden sowie eine Stellvertreterin bzw.\ einen Stellvertreter.
\label{wahl}

Die Amtszeit der Vorstandsmitglieder beträgt ein Jahr. Im Falle des vorzeitigen Ausscheidens
eines Vorstandsmitglieds kann der Konvent gemäß \refPar{wahl} ein Ersatzmitglied für die restliche Amtszeit wählen. Der Vorstand kann in geheimer Wahl mit der 2/3-Mehrheit der anwesenden Mitglieder des Konvents abgewählt werden.

Der Vorstand ist beschlussfähig, wenn mindestens die Hälfte seiner Mitglieder anwesend ist. Für die Beschlussfassung genügt die einfache Mehrheit der anwesenden Vorstandsmitglieder. Bei Stimmengleichheit entscheidet die Stimme des Vorsitzenden.


\Paragraph{title = Sitzungen}
Der Konvent tagt in der Regel einmal pro Semester. 
Eine außerordentliche Sitzung ist einzuberufen, wenn mindestens 10\% aller Mitglieder oder die Hälfte des Vorstands des Konvents dies verlangt.
Die Vorsitzende bzw.\ der Vorsitzende des Vorstands beruft die Sitzungen des Konvents ein.
Die Einladung soll den Mitgliedern spätestens eine Woche vor der Sitzung vorliegen. Ein Versand per E-Mail ist ausreichend. Mit der Einladung ist eine vorläufige Tagesordnung zu versenden.

Der Vorstand stellt die vorläufige Tagesordnung auf. Erster Tagesordnungspunkt ist die Feststellung der mit der Einladung versandten vorläufigen Tagesordnung. Zu Beginn der Sitzung können auf Antrag zusätzliche Tagesordnungspunkte aufgenommen werden.

Die Vorsitzende bzw.\ der Vorsitzende oder ein anderes Vorstandsmitglied leitet die Sitzungen.	

Der Konvent tagt in der Regel nichtöffentlich. Über die KIT-Öffentlichkeit einer Sitzung sowie das Hinzuziehen von Gästen beschließt der Konvent mit der einfachen Mehrheit seiner anwesenden Mitglieder.

Über die Sitzungen des Konvents wird ein Protokoll erstellt und per E-Mail an die Mitglieder verteilt.

Der Konvent kann Arbeitsgruppen einsetzen, die zwischen den Sitzungen des Konvents arbeiten
und den Sitzungen des Konvents berichten.
Jeder Arbeitsgruppe soll ein Mitglied des Vorstands angehören.

\Paragraph{title = Beschlussfassung}
Der Konvent ist beschlussfähig, wenn 11 seiner Mitglieder anwesend sind. Die Sitzungsleiterin bzw.\ der Sitzungsleiter stellt die Beschlussfähigkeit fest.	

Der Konvent fasst Beschlüsse mit der einfachen Mehrheit der anwesenden Mitglieder. Die
Beschlussfassung erfolgt in der Regel durch Handzeichen. Auf Verlangen eines
anwesenden Mitglieds ist ein Beschluss in geheimer Abstimmung zu fassen. 
Entscheidungen in Personalangelegenheiten erfolgen grundsätzlich in geheimer Abstimmung.

\Paragraph{title = Änderung der Geschäftsordnung}
Änderungsanträge zur Geschäftsordnung müssen mit der Einladung angekündigt werden. Sie
bedürfen eines Beschlusses mit 2/3 Mehrheit aller anwesenden Mitglieder des Konvents.

\Paragraph{title = Inkrafttreten}
Die Geschäftsordnung tritt am Tag der Beschlussfassung durch den Konvent in Kraft.

% \enlargethispage*{\baselineskip}
% \pagebreak

\end{contract}
\end{document}

% vim: spelllang=de
